%-------------------------------------------------------------------------------
%	SECTION TITRE
%-------------------------------------------------------------------------------
\cvsection{Intérêts de Recherche}

%-------------------------------------------------------------------------------
%	CONTENU
%-------------------------------------------------------------------------------
\begin{cventries}

    %---------------------------------------------------------
    \cventry
    {Intelligence Artificielle \& Machine Learning}
    {Détection d'Intrusions par Apprentissage Automatique} % Domaine
    {Recherche Active} % Statut
    {} % Dates
    {
        \begin{cvitems} % Description(s)
            \item {\textbf{Problématique :} Développement d'algorithmes ML pour la détection temps réel d'intrusions réseau}
            \item {\textbf{Approche :} Ensemble de modèles (Random Forest, SVM, Neural Networks) avec optimisation des performances}
            \item {\textbf{Résultats actuels :} Précision de 98.1\% sur datasets NSL-KDD et CICIDS avec latence <50ms}
            \item {\textbf{Innovation :} Réduction des faux positifs <2\% par fusion de modèles et analyse comportementale}
        \end{cvitems}
    }

    %---------------------------------------------------------
    \cventry
    {Systèmes Distribués \& Architecture}
    {Sécurité des Systèmes Distribués à Grande Échelle} % Domaine
    {Recherche Appliquée} % Statut
    {} % Dates
    {
        \begin{cvitems} % Description(s)
            \item {\textbf{Focus :} Architecture de sécurité pour infrastructures distribuées et microservices}
            \item {\textbf{Méthodologie :} Approche systémique de la sécurité dans les systèmes ouverts et communicants}
            \item {\textbf{Technologies :} Conteneurisation sécurisée, orchestration, monitoring distribué}
            \item {\textbf{Applications :} SOC distribués, détection collaborative d'anomalies}
        \end{cvitems}
    }

    %---------------------------------------------------------
    \cventry
    {Cybersécurité Opérationnelle}
    {SIEM/SOAR \& Réponse Automatisée aux Incidents} % Domaine
    {Recherche \& Développement} % Statut
    {} % Dates
    {
        \begin{cvitems} % Description(s)
            \item {\textbf{Recherche :} Automatisation intelligente de la réponse aux incidents de sécurité}
            \item {\textbf{Outils étudiés :} Wazuh, TheHive, MISP, Cortex pour orchestration de sécurité}
            \item {\textbf{Innovation :} Intégration d'IA dans les workflows SOAR pour prise de décision autonome}
            \item {\textbf{Impact :} Réduction du temps de réponse aux incidents critiques}
        \end{cvitems}
    }

    %---------------------------------------------------------
\end{cventries}