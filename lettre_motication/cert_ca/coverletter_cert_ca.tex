%!TEX TS-program = xelatex
%!TEX encoding = UTF-8 Unicode
% Awesome CV LaTeX Template for Cover Letter

%-------------------------------------------------------------------------------
% CONFIGURATIONS
%-------------------------------------------------------------------------------
\documentclass[11pt, a4paper]{awesome-cv}
\geometry{left=1.4cm, top=.8cm, right=1.4cm, bottom=1.8cm, footskip=.5cm}
\fontdir[fonts/]
\colorlet{awesome}{awesome-red}
\setbool{acvSectionColorHighlight}{true}
\renewcommand{\acvHeaderSocialSep}{\quad\textbar\quad}

%-------------------------------------------------------------------------------
%	PERSONAL INFORMATION
%-------------------------------------------------------------------------------
\photo[circle,noedge,left]{profile.png}
\name{Mohammed}{Sbihi}
\position{Cybersecurity Engineering Student}
\address{Route Ain Chkef, Fes, Morocco}

\mobile{(+212) 636208830}
\email{mohammedsbihi11@gmail.com}
\github{Med10S}
\linkedin{mohammed-sbihi}
\quote{Passionné de cybersécurité et d'analyse forensique}

%-------------------------------------------------------------------------------
%	LETTER INFORMATION
%-------------------------------------------------------------------------------
\recipient{Équipe de Recrutement}{CERT - Groupe Crédit Agricole\\France}
\letterdate{22 septembre 2025}
\lettertitle{Candidature pour le Stage Cybersécurité - CERT}
\letteropening{Madame, Monsieur,}
\letterclosing{Cordialement,}

%-------------------------------------------------------------------------------
\begin{document}

\makecvheader[R]

\makecvfooter
{22 septembre 2025}
{Mohammed Sbihi~~~·~~~Lettre de Motivation}
{}

\makelettertitle

\begin{cvletter}

    Depuis mon enfance, j'ai toujours été fasciné par les mystères du numérique et les défis que représente la protection des systèmes informatiques. Cette passion m'a naturellement mené vers la cybersécurité, un domaine où chaque jour apporte son lot de nouveaux défis et d'apprentissages.

    Votre offre de stage au CERT du Groupe Crédit Agricole a immédiatement captivé mon attention. L'idée de rejoindre une équipe de "chasseurs de menaces" et de contribuer à la sécurité d'une institution aussi importante me fait vibrer. Ce qui me séduit particulièrement, c'est votre approche 100% Open Source - une philosophie que je partage profondément et que j'applique dans mes projets personnels.

    Actuellement, je consacre mes soirées et week-ends à développer mon propre écosystème SOC, intégrant Wazuh, TheHive, Cortex, MISP et n8n. Ce n'est pas juste un projet académique pour moi - c'est une véritable passion ! Chaque nouvelle fonctionnalité implémentée, chaque workflow automatisé me donne cette satisfaction unique du créateur qui voit son œuvre prendre vie. Ce projet m'a non seulement permis de maîtriser les outils, mais surtout de comprendre la psychologie des attaquants et les subtilités de la défense.

    En tant que président du club SECOPS à l'ENSAF, j'ai découvert ma vocation pour le partage et la transmission. Organiser des CTF, animer des ateliers de sensibilisation, voir l'étincelle dans les yeux d'un étudiant qui comprend enfin un concept complexe... Ces moments me rappellent pourquoi j'ai choisi cette voie.

    Ce qui m'attire le plus dans votre mission, c'est cette promesse d'immersion totale dans le DFIR et la Threat Intelligence. Je rêve de ces moments où, face à un incident complexe, l'équipe unit ses forces pour dénouer les fils d'une attaque sophistiquée. L'idée d'apprendre auprès d'experts reconnus, de contribuer concrètement à la sécurité numérique de millions d'utilisateurs, représente pour moi bien plus qu'un stage - c'est un tremplin vers la carrière dont je rêve.

    Je suis prêt à m'investir corps et âme dans cette aventure. Ma curiosité insatiable, ma détermination à toute épreuve et ma passion sincère pour la cybersécurité feront, j'en suis convaincu, de moi un élément moteur de votre équipe.

    J'ai hâte de pouvoir vous rencontrer et de partager avec vous ma vision de la cybersécurité et mes ambitions pour l'avenir.

\end{cvletter}

\makeletterclosing

\end{document}