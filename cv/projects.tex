%-------------------------------------------------------------------------------
%	SECTION TITRE
%-------------------------------------------------------------------------------
\cvsection{Projets}

%-------------------------------------------------------------------------------
%	CONTENU
%-------------------------------------------------------------------------------
\begin{cventries}

  %---------------------------------------------------------
  \cventry
  {IA - Cybersécurité - Architecture Distribuée}
  {Système de Détection d'Intrusion Distribué (IDS)} % Projet
  {ENSAF - Projet de Fin d'Année} % Lieu
  {Janvier 2025 - Juillet 2025} % Dates
  {
    \begin{cvitems} % Description(s)
      \item {
                  \textbf{Intelligence Artificielle :}
                  Développement d'un ensemble de modèles ML (KNN, MLP, XGBoost) avec 98.1\% de précision pour la détection d'intrusions réseau.}
      \item {
                  \textbf{Architecture Microservices :}
                  Conception d'un système distribué temps réel avec capture de paquets, extraction de features UNSW-NB15 et API FastAPI.}
      \item {
                  \textbf{Technologies Avancées :}
                  Stack complète Python, Docker, Redis, Prometheus avec analyse de 9 types d'attaques (DoS, Reconnaissance, Exploitation, etc.).}
      \item {
                  \textbf{Performance :}
                  Système capable de traiter 1000+ prédictions/seconde avec latence <50ms et taux de faux positifs <2\%.}
    \end{cvitems}
  }
\end{cventries}
\begin{cventries}
  %---------------------------------------------------------
  \cventry
  {Machine Learning - Cybersécurité}
  {Détection d'Anomalies de Connexion avec Java et Isolation Forest} % Projet
  {ENSAF} % Lieu
  {Nov. 2024 - Déc. 2024} % Dates
  {
    \begin{cvitems} % Description(s)
      \item {
                  \textbf{Développement Backend :}
                  Génération et simulation de logs en Java.}
      \item {
                  \textbf{Machine Learning :}
                  Détection des connexions anormales à l’aide d’Isolation Forest.}
      \item {
                  \textbf{Cybersécurité :}
                  Identification des accès suspects en dehors des horaires habituels.}
      \item {
                  \textbf{Big Data - Analyse :}
                  Traitement et analyse des logs pour la détection d’anomalies.}
    \end{cvitems}
  }
\end{cventries}
\begin{cventries}
  %---------------------------------------------------------
  \cventry
  {DevOps - Cybersécurité}
  {Simulation d’un SOC} % Projet
  {ENSAF} % Lieu
  {Février 2025 - Mars 2025} % Dates
  {
    \begin{cvitems} % Description(s)
      \item {
                  \textbf{Cybersécurité :}
                  Détection d’attaques (DDoS, scans, intrusions) avec Suricata.}
      \item {
                  \textbf{DevOps \& Conteneurisation :}
                  Déploiement automatisé sur GNS3 avec Docker.}
      \item {
                  \textbf{Virtualisation \& Réseaux :}
                  Simulation d’un réseau sécurisé.}
      \item {
                  \textbf{Big Data \& SIEM :}
                  Analyse et visualisation des logs avec l’ELK Stack.}
    \end{cvitems}
  }

  %---------------------------------------------------------
\end{cventries}
