%-------


%	SECTION TITRE
%-------------------------------------------------------------------------------
\cvsection{Expérience Professionnelle}

%-------------------------------------------------------------------------------
%	CONTENU
%-------------------------------------------------------------------------------
\begin{cventries}

  %---------------------------------------------------------
  \cventry
  {Stagiaire en Ingénierie Cybersécurité (Projet SOC)} % Intitulé du poste
  {Centre Hospitalier Universitaire (CHU) de Fès} % Organisation
  {Fès, Maroc} % Lieu
  {Juin 2025 - Août 2025} % Dates
  {
    \begin{cvitems}
      \item {
                  \textbf{Conception et déploiement de A à Z d'un Security Operations Center (SOC)} pour la surveillance d'une infrastructure hospitalière critique.}
      \item {
                  \textbf{Intégration d'une stack SIEM/SOAR complète} incluant Wazuh pour la détection, TheHive pour la gestion des cas, Cortex pour l'analyse et MISP pour le renseignement sur les menaces.}
      \item {
                  \textbf{Automatisation avancée des workflows de réponse aux incidents} via n8n (Node.js) et des scripts Bash, réduisant les temps de réaction manuels.}
      \item {
                  \textbf{Développement de preuves de concept (PoC) d'attaques} (ex: EternalBlue) en Ruby et C pour valider et renforcer l'efficacité des règles de détection de l'IDS (Suricata).}
    \end{cvitems}
  }
  %---------------------------------------------------------
  \cventry
  {Stage en Architecture Réseau} % Intitulé du poste
  {SNRT} % Organisation
  {Rabat, Maroc} % Lieu
  {Juillet 2024 - Août 2024} % Dates
  {
    \begin{cvitems} % Description(s) des tâches et responsabilités
      \item {Réalisation d'un projet de topologie réseau avec GNS3, permettant la modélisation et l'analyse de divers scénarios de sécurité.}
      \item {Simulation d’une attaque par Déni de Service (DoS) pour évaluer la résilience du réseau.}
      \item {Optimisation des dispositifs pare-feu et des politiques de sécurité afin de renforcer la protection contre les cybermenaces.}
    \end{cvitems}
  }
  %---------------------------------------------------------
\end{cventries}
